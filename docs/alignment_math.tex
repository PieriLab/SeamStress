\documentclass[11pt,a4paper]{article}
\usepackage[margin=1in]{geometry}
\usepackage{amsmath}
\usepackage{amssymb}
\usepackage{algorithm}
\usepackage{algpseudocode}
\usepackage{graphicx}
\usepackage{hyperref}
\usepackage{enumerate}
\usepackage{mathtools}
\usepackage{bm}

% Define argmin operator
\DeclareMathOperator*{\argmin}{arg\,min}

\title{Mathematical Framework for Molecular Alignment in SeamStress}
\author{SeamStress Documentation}
\date{\today}

\begin{document}

\maketitle

\begin{abstract}
This document provides a comprehensive mathematical description of the alignment algorithms used in SeamStress for molecular geometry alignment and analysis. We describe the Kabsch algorithm, weighted alignment schemes, permutation search strategies, and the two-stage alignment process used for optimal molecular superposition.
\end{abstract}

\tableofcontents
\newpage

\section{Introduction}

SeamStress aligns molecular geometries using the Kabsch algorithm combined with optimal atom permutation search. The workflow involves:

\begin{enumerate}
\item \textbf{Connectivity grouping}: Molecules are grouped by connectivity (SMILES hash)
\item \textbf{Permutation search}: Finding optimal atom correspondence between molecules
\item \textbf{Kabsch alignment}: Optimal rotation and translation for superposition
\item \textbf{Heavy atom weighting}: Optional prioritization of heavy atoms in alignment
\item \textbf{RMSD calculation}: Quantifying structural similarity
\end{enumerate}

\section{The Kabsch Algorithm}

\subsection{Standard Kabsch Algorithm}

The Kabsch algorithm finds the optimal rotation matrix $\mathbf{R}$ and translation vector $\mathbf{t}$ to align two sets of points.

\subsubsection{Problem Statement}

Given two sets of $N$ points:
\begin{itemize}
\item Reference coordinates: $\mathbf{P} = \{\mathbf{p}_1, \mathbf{p}_2, \ldots, \mathbf{p}_N\} \in \mathbb{R}^{N \times 3}$
\item Target coordinates: $\mathbf{Q} = \{\mathbf{q}_1, \mathbf{q}_2, \ldots, \mathbf{q}_N\} \in \mathbb{R}^{N \times 3}$
\end{itemize}

Find rotation $\mathbf{R} \in SO(3)$ and translation $\mathbf{t} \in \mathbb{R}^3$ that minimize:
\begin{equation}
\text{RMSD} = \sqrt{\frac{1}{N} \sum_{i=1}^{N} \|\mathbf{p}_i - (\mathbf{R}\mathbf{q}_i + \mathbf{t})\|^2}
\end{equation}

\subsubsection{Algorithm Steps}

\textbf{Step 1: Compute centroids}
\begin{equation}
\bar{\mathbf{p}} = \frac{1}{N}\sum_{i=1}^{N} \mathbf{p}_i, \quad \bar{\mathbf{q}} = \frac{1}{N}\sum_{i=1}^{N} \mathbf{q}_i
\end{equation}

\textbf{Step 2: Center the coordinates}
\begin{equation}
\mathbf{P}' = \mathbf{P} - \bar{\mathbf{p}}, \quad \mathbf{Q}' = \mathbf{Q} - \bar{\mathbf{q}}
\end{equation}

\textbf{Step 3: Compute the covariance matrix}
\begin{equation}
\mathbf{H} = {\mathbf{P}'}^T \mathbf{Q}'
\end{equation}

\textbf{Step 4: Singular Value Decomposition (SVD)}
\begin{equation}
\mathbf{H} = \mathbf{U} \mathbf{\Sigma} \mathbf{V}^T
\end{equation}

\textbf{Step 5: Compute optimal rotation}
\begin{equation}
\mathbf{R} = \mathbf{V} \mathbf{U}^T
\end{equation}

If $\det(\mathbf{R}) < 0$ (reflection), correct by flipping the sign of the last column of $\mathbf{V}$:
\begin{equation}
\mathbf{V}[:, -1] \leftarrow -\mathbf{V}[:, -1], \quad \mathbf{R} = \mathbf{V} \mathbf{U}^T
\end{equation}

\textbf{Step 6: Apply transformation}

The aligned coordinates are:
\begin{equation}
\mathbf{Q}_{\text{aligned}} = (\mathbf{Q} - \bar{\mathbf{q}}) \mathbf{R} + \bar{\mathbf{p}}
\end{equation}

\subsection{Weighted Kabsch Algorithm}

For molecular alignment, different atoms should contribute differently based on their atomic mass or type.

\subsubsection{Weight Definition}

Define weights $w_i$ for each atom $i$. In SeamStress, three weight schemes are available:

\begin{enumerate}
\item \textbf{Mass weighting} (default): $w_i = m_i \cdot f_i$ where
\begin{equation}
m_i = \text{atomic mass}, \quad f_i = \begin{cases}
h & \text{if atom } i \text{ is heavy} \\
1 & \text{if atom } i \text{ is hydrogen}
\end{cases}
\end{equation}
Here $h$ is the \texttt{heavy\_atom\_factor} (default: $h = 1.0$).

\item \textbf{Uniform weighting}: $w_i = 1$ for all atoms

\item \textbf{Heavy-only weighting}: $w_i = \begin{cases}
1 & \text{if atom } i \text{ is not H} \\
0 & \text{if atom } i \text{ is H}
\end{cases}$
\end{enumerate}

\subsubsection{Weighted Algorithm}

\textbf{Step 1: Normalize weights}
\begin{equation}
w_i' = \frac{w_i}{\sum_{j=1}^{N} w_j}
\end{equation}

\textbf{Step 2: Weighted centroids}
\begin{equation}
\bar{\mathbf{p}} = \sum_{i=1}^{N} w_i' \mathbf{p}_i, \quad \bar{\mathbf{q}} = \sum_{i=1}^{N} w_i' \mathbf{q}_i
\end{equation}

\textbf{Step 3: Center coordinates}
\begin{equation}
\mathbf{P}' = \mathbf{P} - \bar{\mathbf{p}}, \quad \mathbf{Q}' = \mathbf{Q} - \bar{\mathbf{q}}
\end{equation}

\textbf{Step 4: Weighted covariance matrix}

Let $\mathbf{W} = \text{diag}(w_1', w_2', \ldots, w_N')$. The covariance matrix is:
\begin{equation}
\mathbf{H} = {\mathbf{P}'}^T \mathbf{W} \mathbf{Q}'
\end{equation}

This can be computed efficiently as:
\begin{equation}
\mathbf{H} = (\sqrt{\mathbf{W}} \mathbf{P}')^T (\sqrt{\mathbf{W}} \mathbf{Q}')
\end{equation}

where $\sqrt{\mathbf{W}} = \text{diag}(\sqrt{w_1'}, \sqrt{w_2'}, \ldots, \sqrt{w_N'})$.

\textbf{Steps 5-6}: Proceed with SVD and rotation computation as in standard Kabsch algorithm.

\section{RMSD Calculation}

The Root Mean Square Deviation (RMSD) quantifies the structural difference after optimal alignment.

\subsection{Definition}

After aligning $\mathbf{Q}$ to $\mathbf{P}$, the RMSD is:
\begin{equation}
\text{RMSD} = \sqrt{\frac{1}{N} \sum_{i=1}^{N} \|\mathbf{p}_i - \mathbf{q}_i^{\text{aligned}}\|^2}
\end{equation}

Expanding the Euclidean norm:
\begin{equation}
\text{RMSD} = \sqrt{\frac{1}{N} \sum_{i=1}^{N} [(p_{i,x} - q_{i,x}^{\text{aligned}})^2 + (p_{i,y} - q_{i,y}^{\text{aligned}})^2 + (p_{i,z} - q_{i,z}^{\text{aligned}})^2]}
\end{equation}

\subsection{Implementation}

In matrix form:
\begin{equation}
\mathbf{D} = \mathbf{P} - \mathbf{Q}_{\text{aligned}}
\end{equation}
\begin{equation}
\text{RMSD} = \sqrt{\frac{1}{N} \sum_{i=1}^{N} \sum_{j=1}^{3} D_{ij}^2} = \sqrt{\frac{1}{N} \|\mathbf{D}\|_F^2}
\end{equation}

where $\|\cdot\|_F$ is the Frobenius norm.

\section{Permutation Search}

For symmetric molecules, finding the optimal atom correspondence is crucial.

\subsection{Problem Statement}

Given molecules with identical connectivity, find permutation $\pi: \{1, \ldots, N\} \to \{1, \ldots, N\}$ that minimizes:
\begin{equation}
\text{RMSD}_{\pi} = \text{RMSD}(\mathbf{P}, \mathbf{Q}_{\pi})
\end{equation}

where $\mathbf{Q}_{\pi}$ applies permutation $\pi$ to rows of $\mathbf{Q}$:
\begin{equation}
\mathbf{Q}_{\pi} = \begin{bmatrix} \mathbf{q}_{\pi(1)} \\ \mathbf{q}_{\pi(2)} \\ \vdots \\ \mathbf{q}_{\pi(N)} \end{bmatrix}
\end{equation}

\subsection{Brute Force Search Algorithm}

SeamStress uses a factored brute force search by atom type.

\subsubsection{Factorization by Atom Type}

Separate heavy atoms (C, N, O, etc.) from hydrogens:
\begin{itemize}
\item Heavy atom indices: $\mathcal{H} = \{i : \text{atom}_i \neq \text{H}\}$, $|\mathcal{H}| = n_h$
\item Hydrogen indices: $\mathcal{I} = \{i : \text{atom}_i = \text{H}\}$, $|\mathcal{I}| = n_H$
\end{itemize}

Total permutations to test: $n_h! \times n_H!$

For ethylene ($\text{C}_2\text{H}_4$): $2! \times 4! = 2 \times 24 = 48$ permutations

\begin{algorithm}
\caption{Standard Permutation Search}
\begin{algorithmic}[1]
\State $\text{RMSD}_{\min} \leftarrow \infty$
\State $\pi_{\text{best}} \leftarrow \text{identity}$
\For{each heavy atom permutation $\pi_h \in S_{n_h}$}
    \For{each hydrogen permutation $\pi_H \in S_{n_H}$}
        \State Combine: $\pi \leftarrow \pi_h \cup \pi_H$
        \State $\mathbf{Q}' \leftarrow \mathbf{Q}_{\pi}$ \Comment{Apply permutation}
        \State $\mathbf{Q}_{\text{aligned}} \leftarrow \text{KabschAlign}(\mathbf{P}, \mathbf{Q}')$
        \State $r \leftarrow \text{RMSD}(\mathbf{P}, \mathbf{Q}_{\text{aligned}})$
        \If{$r < \text{RMSD}_{\min}$}
            \State $\text{RMSD}_{\min} \leftarrow r$
            \State $\pi_{\text{best}} \leftarrow \pi$
        \EndIf
    \EndFor
\EndFor
\State \Return $(\pi_{\text{best}}, \text{RMSD}_{\min})$
\end{algorithmic}
\end{algorithm}

\section{Fragment-Based Permutation Optimization}

For molecules where each heavy atom has exactly one bonded hydrogen (e.g., benzene), we can treat heavy atom-hydrogen pairs as rigid fragments.

\subsection{Applicability Condition}

Fragment mode applies when:
\begin{equation}
\forall i \in \mathcal{H}: |\{j \in \mathcal{I} : \text{bonded}(i, j)\}| = 1
\end{equation}

This means each heavy atom has exactly one bonded hydrogen.

\subsection{Fragment Definition}

Define fragments $F_k$ as heavy atom + bonded hydrogen pairs:
\begin{equation}
F_k = \{h_k, H_k\}
\end{equation}

where $h_k$ is a heavy atom and $H_k$ is its bonded hydrogen.

\subsection{Complexity Reduction}

\textbf{Standard mode} (benzene with 6 carbons, 6 hydrogens):
\begin{equation}
\text{Permutations} = 6! \times 6! = 720 \times 720 = 518{,}400
\end{equation}

\textbf{Fragment mode} (benzene with 6 C-H fragments):
\begin{equation}
\text{Permutations} = 6! = 720
\end{equation}

\textbf{Speedup}: $\frac{518{,}400}{720} = 720\times$ faster!

\subsection{Fragment Permutation Algorithm}

\begin{algorithm}
\caption{Fragment-Based Permutation Search}
\begin{algorithmic}[1]
\State Build fragment map: $F = \{F_1, F_2, \ldots, F_{n_h}\}$
\State $\text{RMSD}_{\min} \leftarrow \infty$
\State $\pi_{\text{best}} \leftarrow \text{identity}$
\For{each fragment permutation $\sigma \in S_{n_h}$}
    \State Initialize $\pi \leftarrow [0, 0, \ldots, 0]$ of length $N$
    \For{$k = 1$ to $n_h$}
        \State $F_{\text{ref}} \leftarrow F_k$ \Comment{Reference fragment}
        \State $F_{\text{tgt}} \leftarrow F_{\sigma(k)}$ \Comment{Target fragment}
        \For{atom $a$ in $F_{\text{ref}}$, atom $b$ in $F_{\text{tgt}}$}
            \State $\pi[a] \leftarrow b$ \Comment{Map atoms in fragments}
        \EndFor
    \EndFor
    \State $\mathbf{Q}' \leftarrow \mathbf{Q}_{\pi}$
    \State $\mathbf{Q}_{\text{aligned}} \leftarrow \text{KabschAlign}(\mathbf{P}, \mathbf{Q}')$
    \State $r \leftarrow \text{RMSD}(\mathbf{P}, \mathbf{Q}_{\text{aligned}})$
    \If{$r < \text{RMSD}_{\min}$}
        \State $\text{RMSD}_{\min} \leftarrow r$
        \State $\pi_{\text{best}} \leftarrow \pi$
    \EndIf
\EndFor
\State \Return $(\pi_{\text{best}}, \text{RMSD}_{\min})$
\end{algorithmic}
\end{algorithm}

\section{Two-Stage Alignment Process}

SeamStress uses a two-stage alignment to separate permutation search from heavy atom weighting.

\subsection{Rationale}

\begin{enumerate}
\item \textbf{Stage 1 (Permutation search)}: Find optimal atom correspondence using mass-weighted alignment
\item \textbf{Stage 2 (Heavy atom refinement)}: Re-align with increased heavy atom weighting using the permutation from Stage 1
\end{enumerate}

This separates the combinatorial optimization (permutation) from the geometric optimization (alignment).

\subsection{Mathematical Formulation}

\textbf{Stage 1: Find optimal permutation}

Use mass-weighted Kabsch with $h = 1.0$:
\begin{equation}
\pi^* = \argmin_{\pi \in S_N} \text{RMSD}_{\text{mass}}(\mathbf{P}, \mathbf{Q}_\pi)
\end{equation}

\textbf{Stage 2: Refine alignment}

Apply heavy atom weighting with $h > 1.0$ (e.g., $h = 10.0$):
\begin{equation}
\mathbf{Q}_{\text{final}} = \text{WeightedKabsch}(\mathbf{P}, \mathbf{Q}_{\pi^*}, h)
\end{equation}

\subsection{Algorithm}

\begin{algorithm}
\caption{Two-Stage Alignment}
\begin{algorithmic}[1]
\State \textbf{Input:} Reference $\mathbf{P}$, Target $\mathbf{Q}$, heavy factor $h$
\State
\State \textbf{// Stage 1: Permutation Search}
\State $\pi^* \leftarrow \text{FindBestPermutation}(\mathbf{P}, \mathbf{Q}, h=1.0)$
\State $\mathbf{Q}' \leftarrow \mathbf{Q}_{\pi^*}$ \Comment{Apply best permutation}
\State
\State \textbf{// Stage 2: Heavy Atom Refinement}
\If{$h > 1.0$}
    \State $\mathbf{Q}_{\text{aligned}} \leftarrow \text{WeightedKabsch}(\mathbf{P}, \mathbf{Q}', h)$
    \State $\text{RMSD} \leftarrow \text{ComputeRMSD}(\mathbf{P}, \mathbf{Q}_{\text{aligned}})$
\Else
    \State $\mathbf{Q}_{\text{aligned}} \leftarrow \mathbf{Q}'$
    \State $\text{RMSD} \leftarrow \text{RMSD from Stage 1}$
\EndIf
\State
\State \Return $(\pi^*, \mathbf{Q}_{\text{aligned}}, \text{RMSD})$
\end{algorithmic}
\end{algorithm}

\section{Complete Alignment Workflows}

\subsection{Mode 1: Multi-Family Alignment}

This mode groups molecules by connectivity and aligns each family independently.

\subsubsection{Workflow}

\begin{enumerate}
\item \textbf{Read geometries}: Load all XYZ files
\item \textbf{Connectivity analysis}: Compute SMILES hash for each molecule
\item \textbf{Family grouping}: Group molecules by SMILES
\item \textbf{Inter-family alignment}: Align family centroids to master reference
\begin{equation}
\mathbf{C}_i^{\text{aligned}} = \text{WeightedKabsch}(\mathbf{C}_{\text{master}}, \mathbf{C}_i, h_{\text{inter}})
\end{equation}
where $h_{\text{inter}}$ is \texttt{inter\_family\_heavy\_atom\_factor}

\item \textbf{Intra-family alignment}: For each family $i$ and molecule $j$:
\begin{equation}
\pi_{ij}^*, \mathbf{M}_{ij}^{\text{aligned}} = \text{TwoStageAlign}(\mathbf{C}_i^{\text{aligned}}, \mathbf{M}_{ij}, h_{\text{intra}})
\end{equation}
where $h_{\text{intra}}$ is \texttt{intra\_family\_heavy\_atom\_factor}
\end{enumerate}

\subsection{Mode 2: Align-All-to-Centroid}

This mode treats all molecules as one family and aligns to a single reference.

\subsubsection{Workflow}

\begin{enumerate}
\item \textbf{Load reference centroid}: Read specified centroid file $\mathbf{C}_{\text{ref}}$
\item \textbf{Align all spawning points}: For each molecule $j$:
\begin{equation}
\pi_j^*, \mathbf{M}_j^{\text{aligned}} = \text{TwoStageAlign}(\mathbf{C}_{\text{ref}}, \mathbf{M}_j, h_{\text{intra}})
\end{equation}

\item \textbf{Align all centroids}: For visualization, align all centroids to reference:
\begin{equation}
\mathbf{C}_k^{\text{aligned}} = \text{KabschAlign}(\mathbf{C}_{\text{ref}}, \mathbf{C}_k)
\end{equation}
No permutation search for centroid alignment (identity permutation only)

\item \textbf{Save for analysis}:
\begin{itemize}
\item Aligned spawns $\to$ \texttt{family\_1/*.xyz}
\item All aligned centroids $\to$ \texttt{family\_1/centroids.xyz} (multi-frame)
\end{itemize}
\end{enumerate}

\subsection{Visualization in Dimensionality Reduction}

\textbf{Mode 1}: Each family centroid plotted as one star ($\star$)
\begin{equation}
\text{Stars} = \{\mathbf{C}_1^{\text{aligned}}, \mathbf{C}_2^{\text{aligned}}, \ldots, \mathbf{C}_{n_{\text{families}}}^{\text{aligned}}\}
\end{equation}

\textbf{Mode 2}: All aligned centroids plotted as stars ($\star$)
\begin{equation}
\text{Stars} = \{\mathbf{C}_1^{\text{aligned}}, \mathbf{C}_2^{\text{aligned}}, \ldots, \mathbf{C}_{n_{\text{centroids}}}^{\text{aligned}}\}
\end{equation}

In both modes, individual spawning points plotted as dots ($\bullet$).

\section{Computational Complexity}

\subsection{Kabsch Algorithm}
\begin{itemize}
\item Centroid computation: $O(N)$
\item Covariance matrix: $O(N)$
\item SVD of $3 \times 3$ matrix: $O(1)$
\item Apply transformation: $O(N)$
\item \textbf{Total}: $O(N)$ where $N$ is number of atoms
\end{itemize}

\subsection{Permutation Search}

\textbf{Standard mode}:
\begin{equation}
\text{Complexity} = n_h! \times n_H! \times O(N)
\end{equation}

\textbf{Fragment mode}:
\begin{equation}
\text{Complexity} = n_h! \times O(N)
\end{equation}

For benzene ($n_h = 6$, $n_H = 6$, $N = 12$):
\begin{itemize}
\item Standard: $720 \times 720 \times O(12) \approx 6.2 \times 10^6$ operations
\item Fragment: $720 \times O(12) \approx 8.6 \times 10^3$ operations
\item Speedup: $720\times$
\end{itemize}

\subsection{Complete Workflow}

For $M$ molecules with $F$ families:

\textbf{Mode 1 (Multi-family)}:
\begin{equation}
O(F \cdot (\text{permutation search}) + M \cdot (\text{permutation search}))
\end{equation}

\textbf{Mode 2 (Align-all-to-centroid)}:
\begin{equation}
O(M \cdot (\text{permutation search}) + C \cdot O(N))
\end{equation}
where $C$ is number of centroids (no permutation search for centroid alignment).

\section{Numerical Stability}

\subsection{Weight Normalization}

Weights are always normalized to sum to 1:
\begin{equation}
\sum_{i=1}^{N} w_i' = 1
\end{equation}

This prevents numerical overflow/underflow issues.

\subsection{SVD Stability}

The SVD is numerically stable and works correctly even for:
\begin{itemize}
\item Nearly degenerate configurations (collinear points)
\item Large variations in coordinate magnitudes
\item Ill-conditioned covariance matrices
\end{itemize}

\subsection{Reflection Detection}

Checking $\det(\mathbf{R}) < 0$ prevents reflections:
\begin{itemize}
\item If $\det(\mathbf{R}) = +1$: proper rotation ✓
\item If $\det(\mathbf{R}) = -1$: reflection detected, corrected by flipping last singular vector
\end{itemize}

\section{Implementation Notes}

\subsection{Heavy Atom Factor Selection}

\textbf{Default} ($h = 1.0$): Mass-weighted only
\begin{itemize}
\item C (mass 12) has $12\times$ influence of H (mass 1)
\item Balanced for most molecules
\end{itemize}

\textbf{Moderate} ($h = 5.0$ to $h = 10.0$): Enhanced heavy atom weighting
\begin{itemize}
\item C has $60\times$ to $120\times$ influence of H
\item Useful when hydrogens cause alignment issues
\item Recommended for inter-family centroid alignment
\end{itemize}

\textbf{Extreme} ($h = 100.0$): Near heavy-only alignment
\begin{itemize}
\item C has $1200\times$ influence of H
\item Essentially ignores hydrogens
\item Use with caution
\end{itemize}

\subsection{RMSD Warning Thresholds}

\textbf{Mode 2 (Align-all-to-centroid)}:
\begin{itemize}
\item Mean RMSD $> 1.0$ Å: Warning that molecules may have different connectivity
\item Individual RMSD $> 0.5$ Å: Flagged as high deviation
\end{itemize}

These thresholds indicate potential issues with the alignment assumption.

\section{References}

\begin{enumerate}
\item Kabsch, W. (1976). "A solution for the best rotation to relate two sets of vectors." \textit{Acta Crystallographica Section A} 32(5): 922-923.

\item Kabsch, W. (1978). "A discussion of the solution for the best rotation to relate two sets of vectors." \textit{Acta Crystallographica Section A} 34(5): 827-828.

\item Coutsias, E.A., Seok, C., \& Dill, K.A. (2004). "Using quaternions to calculate RMSD." \textit{Journal of Computational Chemistry} 25(15): 1849-1857.

\item Theobald, D.L. (2005). "Rapid calculation of RMSDs using a quaternion-based characteristic polynomial." \textit{Acta Crystallographica Section A} 61(4): 478-480.
\end{enumerate}

\end{document}
